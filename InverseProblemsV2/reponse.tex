\documentclass[12pt]{article}
\pdfoutput=1

\usepackage[T1]{fontenc}
\usepackage{verbatim}
\usepackage{float}
\usepackage{amsthm}
\usepackage{amsmath}
\usepackage{amssymb}
\usepackage{graphicx}
\usepackage{color}
\usepackage{url}
\usepackage{caption}
\usepackage{subcaption}
\usepackage{mathtools} 
\usepackage{stackrel} 

\newcommand{\LL}{\mathcal{L}}
\newcommand{\E}{\mathbb{E}}
\newcommand{\I}{\mathcal{I}}
\newcommand{\ep}{\varepsilon}
\newcommand{\Z}{\mathbb{Z}}
\newcommand{\GCD}{\mathbf{GCD}}
\newcommand{\XX}{\mathcal{X}}
\newcommand{\SUM}{\text{sum}}
\newcommand{\1}{\mathbf{1}}
\newcommand{\rr}{\textbf{r}}
\newcommand{\ii}{\textbf{i}}
\newcommand{\jj}{\textbf{j}}
\newcommand{\Poisson}{\text{Poisson}}
\newcommand{\II}{\mathcal{I}}
\newcommand{\kk}{\textbf{k}}
\newcommand{\RR}{\mathbb{R}}
\newcommand{\mb}{\mathbf}
\newcommand{\mk}{\mathfrak}
\newcommand{\mc}{\mathcal}
\newcommand{\TODO}[1]{{\color{red}{[#1]}}}
\newcommand{\revised}[1]{{\color{magenta}{#1}}}
\newcommand{\R}{\mathbb{R}}
%\newcommand{\M}{\mathcal{M}}
\newcommand{\M}{m}
\renewcommand{\P}{\mathbb{P}}
\renewcommand{\L}{\mathcal{L}}

\makeatletter

\newcommand{\reals}{\mathbb{R}}
\newcommand{\RL}{\mathbb{R}^L}
\newcommand{\tamir}{x}
\newcommand{\CL}{\mathbb{C}^L}
\newcommand{\RN}{\mathbb{R}^N}
\newcommand{\RNN}{\mathbb{R}^{N\times N}}
\newcommand{\RPP}{\mathbb{R}^{P\times P}}
\newcommand{\CNN}{\mathbb{C}^{N\times N}}
\newcommand{\inner}[1]{\left\langle {#1} \right\rangle}
\newcommand{\hx}{\hat{x}} 
\newcommand{\one}{\mathbf{1}} 
\newcommand{\be}
{\begin{equation}}
\newcommand{\ee}
{\end{equation}}
%\renewcommand{\P}{\mathbb{P}}
\newcommand{\aseq}{\stackrel{a.s.}{=}}
\renewcommand{\P}{\mathrm{Prob}}


\theoremstyle{plain}
\newtheorem{thm}{\protect\theoremname}[section]
\theoremstyle{definition}
\newtheorem{defn}[thm]{\protect\definitionname}
\theoremstyle{remark}
\newtheorem{claim}[thm]{\protect\claimname}
\theoremstyle{plain}
\newtheorem{lem}[thm]{\protect\lemmaname}
\newtheorem*{lem*}{Lemma}
\theoremstyle{remark}
\newtheorem{rem}[thm]{\protect\remarkname}
\theoremstyle{plain}
\newtheorem{corollary}[thm]{\protect\corollaryname}
\theoremstyle{plain}
\newtheorem{conjecture}[thm]{\protect\conjecturename}
\theoremstyle{plain}
\newtheorem{proposition}[thm]{\protect\propositionname}
\providecommand{\claimname}{Claim}
\providecommand{\definitionname}{Definition}
\providecommand{\lemmaname}{Lemma}
\providecommand{\remarkname}{Remark}
\providecommand{\theoremname}{Theorem}
\providecommand{\corollaryname}{Corollary}
\providecommand{\propositionname}{Proposition}
\providecommand{\conjecturename}{Conjecture}

\usepackage{authblk}
\renewcommand*{\Affilfont}{\normalsize}
\setlength{\affilsep}{2em}   % set the space between author and affiliation

\usepackage[margin=3cm]{geometry}

\allowdisplaybreaks
\numberwithin{equation}{section}


\RequirePackage[colorlinks,citecolor=blue,urlcolor=blue,linkcolor=blue]{hyperref}


\begin{document}

%\begin{frontmatter}


\title{Response to referees: Multi-target detection with application to cryo-electron microscopy}

\author{Tamir Bendory, Nicolas Boumal, Will Leeb, Eitan Levin, and Amit Singer}

\date{}
\maketitle

We thank the editor and the referees for reading our paper. Below, we provide a detailed, one-to-one response to all comments. 

\section{Referee 1}

\noindent \textbf{Comment:} The only suggestion I have is to include a brief discussion of validation - the experiments here are from simulated data with known signals. If this technology were ever to be adopted in experimental practice, some form of statistical validation will be essential.\\

\noindent \textbf{Response:} The question of statistical validation is of great importance in any scientific field, including single particle reconstruction using cryo-EM. %This question is more general that the specific method applied.
Unfortunately,  there is  no rigorous method to validate a structure that was constituted using cryo-EM. In practice, researchers in the field use several useful heuristics, such as  comparing the elucidated structure with structures that were reconstructed by other imaging modalities (e.g., X-ray crystallography), or different statistical techniques (e.g., expectation-maximization). The validation problem is even more severe in the low SNR regime (the main interest of this paper) in which visual assessment of the data might be impossible. Our method does not offer insights for this question.

While validation is beyond the scope of this paper, autocorrelation analysis provides a flexible framework to describe reach generative models of the data. We added a paragraph in the summary to explain this issue. 
 

\section{Referee 3}

\noindent \textbf{Comment:} An important point in practical image reconstruction, especially in modalities like phase retrieval, is that precise knowledge of the support of the image makes the reconstruction problem considerably easier.  Beyond that, having an image with a sharply delineated, known boundary is, in fact, a huge advantage. Such images are used in the examples in this paper, along with an exact knowledge of their support.  In most realistic imaging applications such information is not available, and this renders these problems inherently more difficult, regardless of the algorithm used. Some mention of this fact should appear in this paper.\\

\noindent \textbf{Response:} Section 4.4 discusses the application of our approach to 2-D images. In this section, we underscore the importance of the knowledge of the exact support for phase retrieval as was shown in [9]. In our setup, if the support cannot be estimated accurately, the solution is to consider the third-order autocorrelation. That being said, as we mention in Section 4, the support of the signal (at least in 1-D) can be determined from the second-order autocorrelation if enough data is provided. \\

\noindent \textbf{Comment:} Though it is clarified in the proof of Prop. 4.5 it is not immediately clear that  the right hand sides of formulae (4.3), (4.4), (4.5) are observable quantities. A remark, after the statement of the proposition, clarifying this  point would be useful.\\

\noindent \textbf{Response:} We added a remark to the proposition to make this point clear.\\

\noindent \textbf{Comment:}  I found it difficult to understand what Figure 2 is demonstrating. Perhaps it is a result of the poor rendering of the grey-scale.
\\ 

\noindent \textbf{Response:} We find this figure important to inspect (empirically) how many signals can be estimated simultaneously from a mix of their autocorrelations. We agree that this figure should be visualized in a colored version of the manuscript. We used a similar figure for the same purpose (but for a different problem) in a previous publication: Boumal, Nicolas, et al. ``Heterogeneous multireference alignment: A single pass approach.'' \emph{2018 52nd Annual Conference on Information Sciences and Systems (CISS)}. IEEE, 2018.



\end{document}


