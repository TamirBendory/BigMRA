\documentclass[english,11pt]{article}

\pdfoutput=1

\usepackage[T1]{fontenc}
\usepackage[latin9]{inputenc}
\usepackage{verbatim}
\usepackage{float}
\usepackage{amsthm}
\usepackage{amsmath}
\usepackage{amssymb}
\usepackage{graphicx}
%\usepackage{multirow}
\usepackage{color}
\usepackage{url}
\usepackage{caption}
\usepackage{subcaption}
\usepackage{mathtools} 
\usepackage[margin=1.2in]{geometry}


\newcommand{\LL}{\mathcal{L}}
\newcommand{\E}{\mathbb{E}}
\newcommand{\I}{\mathcal{I}}
\newcommand{\ep}{\varepsilon}
\newcommand{\Z}{\mathbb{Z}}
\newcommand{\GCD}{\mathbf{GCD}}
\newcommand{\XX}{\mathcal{X}}
\newcommand{\SUM}{\text{sum}}
\newcommand{\1}{\mathbf{1}}


\newcommand{\TODO}[1]{{\color{red}{[#1]}}}

\makeatletter

%%%%%%%%%%%%%%%%%%%%%%%%%%%%%% Textclass specific LaTeX commands.
%\numberwithin{equation}{section}
%\numberwithin{figure}{section}
\theoremstyle{plain}
\newtheorem{thm}{\protect\theoremname}
\theoremstyle{definition}
\newtheorem{defn}[thm]{\protect\definitionname}
\theoremstyle{remark}
\newtheorem{claim}[thm]{\protect\claimname}
\theoremstyle{plain}
\newtheorem{lem}[thm]{\protect\lemmaname}

\newtheorem*{lem*}{Lemma}

\theoremstyle{remark}
\newtheorem{rem}[thm]{\protect\remarkname}
\theoremstyle{plain}
\newtheorem{corollary}[thm]{\protect\corollaryname}
\theoremstyle{plain}
\newtheorem{proposition}[thm]{\protect\propositionname}
%%%%%%%%%%%%%%%%%%%%%%%%%%%%%% User specified LaTeX commands.
%\usepackage{slashbox}

\usepackage{babel}
\providecommand{\claimname}{Claim}
\providecommand{\definitionname}{Definition}
\providecommand{\lemmaname}{Lemma}
\providecommand{\remarkname}{Remark}
\providecommand{\theoremname}{Theorem}
\providecommand{\corollaryname}{Corollary}
\providecommand{\propositionname}{Proposition}


\newcommand{\reals}{\mathbb{R}}
\newcommand{\RL}{\mathbb{R}^L}
\newcommand{\CL}{\mathbb{C}^L}
\newcommand{\RN}{\mathbb{R}^N}
\newcommand{\RNN}{\mathbb{R}^{N\times N}}
\newcommand{\CNN}{\mathbb{C}^{N\times N}}
\newcommand{\inner}[1]{\left\langle {#1} \right\rangle}
\newcommand{\hx}{\hat{x}} 
\newcommand{\one}{\mathbf{1}} 
\newcommand{\SNR}{\ensuremath{\textsf{SNR}}}

\begin{document}
	
	\title{The autocorrelation functions in cryo--EM}
	
	
	\author{Tamir Bendory, Nicolas Boumal, William Leeb and Amit Singer}
	\maketitle



The 3-D Fourier transform of an L-bandlimited 3-D volume (e.g., particle) can be expanded into spherical harmonics:
\begin{equation} \label{eq:volume}
\hat{V}(k,\theta,\phi)  = \sum_{\ell=0}^{L} \sum_{m=-\ell}^{\ell} A_{\ell,m}(k)Y_\ell^{m}(\theta,\phi),
\end{equation}
where $\theta \in [0,\pi)$ is the polar angle, $\phi \in [0,2\pi)$ is the azimuthal angle, $k$ is the radial coordinate, $Y_{\ell}^m(\theta,\phi)$ is the spherical harmonic of degree $\ell$ and order $m$ and  $A_{\ell,m}(k)$ are the associated spherical harmonics coefficients. The goal is to estimate the functions $A_{l,m}$.
A rotation of the volume by $\omega\in SO(3)$ can be described using the Wigner D-function $D_{m,m'}^{\ell}$:
\begin{equation}
\begin{split}
(R_\omega \hat{V})(k,\theta,\phi)  &= \sum_{\ell=0}^{L} \sum_{m=-\ell}^{\ell} A_{\ell,m}(k)(R_\omega Y_\ell^{m})(\theta,\phi) \\
&= \sum_{\ell=0}^{L} \sum_{m=-\ell}^{\ell} A_{\ell,m}(k) 
\sum_{m'=-\ell}^{\ell}{D_{m,m'}^{\ell}(\omega)}Y_\ell^{m'}(\theta,\phi).
\end{split}
\end{equation}

By the Fourier slice theorem, the Fourier transform of each cryo-EM measurement (that is, each projection) is a slice of $\hat V$, associated with $\theta= \pi/2$, after $\hat{V}$ was rotated by $\omega\in SO(3)$. Explicitly, the Fourier transform of a projection from the viewing direction $\omega$ is related to the spherical harmonic coefficients of the object through:
\begin{equation} \label{eq:P}
\begin{split}
\hat P_\omega(k,\phi)  = \sum_{\ell=0}^{L} \sum_{m=-\ell}^{\ell} A_{\ell,m}(k) 
\sum_{m'=-\ell}^{\ell}{D_{m,m'}^{\ell}(\omega)}Y_\ell^{m'}(\pi/2,\phi).
\end{split}
\end{equation}

Next, we relate the projections $P_\omega$ to the mean and the autocorrelation functions, both computable from the observed micrographs.
The mean of the micrograph is proportional to 
\begin{equation}
M_1 \propto \sum_{n=1}^N \sum_{x,y}P_{\omega_n}(x,y),
\end{equation}
where $\omega_n$ denotes the viewing direction of the $n$th projection. By taking $n\to\infty$, we get  
\begin{equation}
M_1 \propto \sum_{x,y}\int_{\omega}P_\omega(x,y)\rho(\omega) d\omega,
\end{equation}
where $\rho(\omega)$ denotes the (possibly unknown) viewing direction distribution over $SO(3)$.

We assume the projections are sufficiently separated so that, in the limit $n\to\infty$, the $(\Delta_x,\Delta_y)$ entry of the second-order autocorrelation of the micrograph is proportional to:
\begin{equation}
M_2(\Delta_x,\Delta_y)\propto\sum_{x,y}\int_{\omega} P_\omega(x,y)P_\omega(x+\Delta_x,y+\Delta_y)\rho(\omega)d\omega  + \textrm{bias}.
\end{equation}
 The assumption here is that $(\Delta_x, \Delta_y)$ are small enough so that, in computing the auto-correlation, points $(x, y)$ and $(x+\Delta_x, y+\Delta_y)$ do not touch distinct particles. 
In the same way and under the same conditions, the third moment is given by 
\begin{equation} 
M_3(\Delta_x^1,\Delta_y^1;\Delta_x^2,\Delta_y^2) \propto \sum_{x,y}\int_{\omega} P_\omega(x,y)P_\omega(x+\Delta_x^1,y+\Delta_y^1)P_\omega(x+\Delta_x^2,y+\Delta_y^2)\rho(\omega)d\omega + \textrm{bias}.
\end{equation}

In order to determine the particle, by~\eqref{eq:volume} one needs to estimate order of $L^3$ spherical harmonics coefficients. If the pixel size is proportional to $1/L$ (to match the volume's resolution), then $M_3$ provides order of $L^4$ equations involving triple products of $P_\omega$. 
However, 
since  the in-plane rotation of each particle image is usually uniformly distributed, $M_3$  depends on only three parameters: the length of the vector $(\Delta_x^1,\Delta_y^1)$, the length of the vector  $(\Delta_x^2,\Delta_y^2)$ and the  angle between the two vectors.  Therefore,  $M_3$  provides only $\sim L^3$ equations. 
 Since $P_\omega$ depends (after coordinate transformation) linearly on the spherical harmonic coefficients through~\eqref{eq:P}, this means we have a system of $\sim L^3$ cubic equations in the $\sim L^3$ sought parameters. Importantly, the coefficients of these equations can be estimated from the micrographs directly, without particle picking.

\end{document}