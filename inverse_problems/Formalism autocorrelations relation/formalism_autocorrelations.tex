%\documentclass{siamart1116}
\documentclass[12pt]{article}

\usepackage[hyphens]{url}
\RequirePackage[colorlinks,citecolor=blue,urlcolor=blue,linkcolor=blue]{hyperref}



\usepackage[ansinew]{inputenc}


%\usepackage{breakcites}


\usepackage{doi}

% Number only equations which are referenced in the text
% The command \eqref needs to be "robustified" in order for
% the numbering to work OK when \eqref is used in moving
% places, such as figure captions. As an alternative, you
% may also prepend eqref with a \protect command, as follows:
% \caption{ ... \protect\eqref{eq:tag} ... }
% No {} around the \eqref command though.
% I actually use the alternative, because the \MakeRobust
% command doens't seem to work.
\usepackage{amsmath}
%\usepackage{fixltx2e,amsmath}
%\MakeRobust{\eqref}
%\usepackage{mathtools}
%\mathtoolsset{showonlyrefs}

% For the jury : number lines -- doesn't work very well though ...
%\usepackage{xcolor}
%\usepackage[displaymath]{lineno}
%% Running line numbers:
%% Same, but that reset on every page:
%\pagewiselinenumbers
%\switchlinenumbers
%% Number only every 5:th line:
%\modulolinenumbers[5]
%\setlength\linenumbersep{5pt}
%\renewcommand\linenumberfont{\normalfont\tiny\sffamily\color{gray}}

\usepackage{mathtools}
\DeclarePairedDelimiter{\ceil}{\lceil}{\rceil}
\DeclarePairedDelimiter{\floorr}{\lfloor}{\rfloor}


\usepackage{amsfonts}
\usepackage{amsthm}  % COMMENT IF SIOPT
\usepackage{amssymb}

\usepackage{color}
\usepackage{bm} % bold math fonts with \bm{math expr}
\usepackage{dsfont} % blackboard bold ones

\usepackage{graphicx}
\usepackage[lmargin=3cm,rmargin=3cm,bottom=3cm,top=3cm]{geometry}



%\usepackage{tikz}
%\definecolor{mycolor1}{rgb}{0.105882,0.619608,0.466667}
%\definecolor{mycolor2}{rgb}{0.85098,0.372549,0.00784314}
%\definecolor{mycolor3}{rgb}{0.458824,0.439216,0.701961}
%\definecolor{mycolor4}{rgb}{0.905882,0.160784,0.541176}
%\definecolor{mycolor5}{rgb}{0.4,0.65098,0.117647}
%\definecolor{mycolor6}{rgb}{0.65098,0.462745,0.113725}
%\definecolor{mycolor7}{rgb}{0.901961,0.670588,0.00784314}
%\definecolor{mycolor8}{rgb}{0.4,0.4,0.4}
%\definecolor{mycolor9}{rgb}{0.301961,0,0.294118}
%\definecolor{mycolor10}{rgb}{0.0313725,0.25098,0.505882}


%\usetikzlibrary{calc}        
%
%\makeatletter
%\newif\ifmygrid@coordinates
%\tikzset{/mygrid/step line/.style={line width=0.80pt,draw=gray!80},
%         /mygrid/steplet line/.style={line width=0.25pt,draw=gray!80}}
%\pgfkeys{/mygrid/.cd,
%         step/.store in=\mygrid@step,
%         steplet/.store in=\mygrid@steplet,
%         coordinates/.is if=mygrid@coordinates}
%\def\mygrid@def@coordinates(#1,#2)(#3,#4){%
%    \def\mygrid@xlo{#1}%
%    \def\mygrid@xhi{#3}%
%    \def\mygrid@ylo{#2}%
%    \def\mygrid@yhi{#4}%
%}
%\newcommand\DrawGrid[3][]{%
%    \pgfkeys{/mygrid/.cd,coordinates=true,step=1,steplet=0.2,#1}%
%    \draw[/mygrid/steplet line] #2 grid[step=\mygrid@steplet] #3;
%    \draw[/mygrid/step line] #2 grid[step=\mygrid@step] #3;
%    \mygrid@def@coordinates#2#3%
%    \ifmygrid@coordinates%
%        \draw[/mygrid/step line]
%        \foreach \xpos in {\mygrid@xlo,...,\mygrid@xhi} {%
%          (\xpos,\mygrid@ylo) -- ++(0,-3pt)
%                              node[anchor=north] {$\xpos$}
%        }
%        \foreach \ypos in {\mygrid@ylo,...,\mygrid@yhi} {%
%          (\mygrid@xlo,\ypos) -- ++(-3pt,0)
%                              node[anchor=east] {$\ypos$}
%        };
%    \fi%
%}
%\makeatother




% use in a table cell to authorize linebreaks with \\ %
% http://tex.stackexchange.com/questions/2441/how-to-add-a-forced-line-break-inside-a-table-cell
%\newcommand{\specialcell}[2][c]{%
%  \begin{tabular}[#1]{@{}#1@{}}#2\end{tabular}}

% for sidewaystable : tables in landscape mode
%\usepackage{rotating}

%\usepackage{psfrag}

\usepackage{algorithm}
\usepackage{algpseudocode}
% Do-While construct
\algdef{SE}[DOWHILE]{Do}{doWhile}{\algorithmicdo}[1]{\algorithmicwhile\ #1}%

% section numbering in margin
%\usepackage{sectsty}
%\makeatletter\def\@seccntformat#1{\protect\makebox[0pt][r]{\csname the#1\endcsname\hspace{12pt}}}\makeatother

\usepackage{mathrsfs} % for \mathscr{ABC} style (curly calligraphy)


%\usepackage[colorlinks=true,bookmarks=false]{hyperref}


% Pick first if SIOPT
%\usepackage[round,compress]{natbib}
\usepackage[square,numbers,compress]{natbib}

%%%%% Activate these in track change mode:
%\newcommand{\add}[1]{{\color{blue}{#1}}}
%\newcommand{\remove}[1]{{\footnote{\color{red}{[removed: #1]}}}}
%\newcommand{\removesafe}[1]{{\scriptsize{\color{red}{[removed: #1]}}}}
%\newcommand{\change}[2]{\remove{#1}\add{#2}}
%\newcommand{\changesafe}[2]{\removesafe{#1}\add{#2}}
%%%%% Activate these in release mode:
%\newcommand{\add}[1]{{#1}}
%\newcommand{\remove}[1]{}
%\newcommand{\removesafe}[1]{}
%\newcommand{\change}[2]{{#2}}
%\newcommand{\changesafe}[2]{{#2}}
%
%\usepackage{subfigure}


% show labels in the margin (for debugging)
% \usepackage[right]{showlabels}



\newcommand{\ER}{{Erd\H{o}s-R\'enyi }}
\newcommand{\argmin}[1]{\underset{#1}{\operatorname{argmin}}}
\newcommand{\argmax}[1]{\underset{#1}{\operatorname{argmax}}}
\newcommand{\transpose}{^\top\! }
\newcommand{\eig}{{\mathrm{eig}}}
\newcommand{\mle}{{\mathrm{MLE}}}
\newcommand{\mlez}{{\mathrm{MLE0}}}
\newcommand{\inner}[2]{\left\langle{#1},{#2}\right\rangle}
\newcommand{\innersmall}[2]{\langle{#1},{#2}\rangle}
\newcommand{\innerbig}[2]{\big\langle{#1},{#2}\big\rangle}
\newcommand{\MSE}{\mathrm{MSE}}
\newcommand{\Egood}{E_\textrm{good}}
\newcommand{\Ebad}{E_\textrm{bad}}
\newcommand{\trace}{\mathrm{Tr}}
\newcommand{\Trace}{\mathrm{Tr}}
\newcommand{\spann}{\mathrm{span}}
\newcommand{\im}{\mathrm{im}}
%\newcommand{\skeww}[1]{\left\lceil #1 \right\rfloor}
\newcommand{\skeww}[1]{\operatorname{skew}\!\left( #1 \right)}
\newcommand{\symmop}{\operatorname{sym}}
\newcommand{\symm}[1]{\symmop\!\left( #1 \right)}
\newcommand{\symmsmall}[1]{\symmop( #1 )}
%\newcommand{\vecc}{\mathrm{vec}}
\newcommand{\veccc}[1]{\vecc\left({#1}\right)}
\newcommand{\sbdop}{\operatorname{sbd}}
\newcommand{\sbd}[1]{\sbdop\!\left({#1}\right)}
\newcommand{\sbdsmall}[1]{\sbdop({#1})}
\newcommand{\RMSE}{\mathrm{RMSE}}


%\newcommand{\Stdpm}{\St(d, p)^m} %%%%%%%%%%%%%%%%%%%%%%%%%%%%%%%%%%%%%%%%%%%%%%
\newcommand{\Stdpm}{\mathcal{M}}


\newcommand{\Proj}{\mathrm{Proj}}
\newcommand{\ProjH}{\mathrm{Proj}^h}
\newcommand{\ProjV}{\mathrm{Proj}^v}
\newcommand{\Exp}{\mathrm{Exp}}
\newcommand{\Log}{\mathrm{Log}}
\newcommand{\expect}{\mathbb{E}}
\newcommand{\expectt}[1]{\mathbb{E}\left\{{#1}\right\}}
%\newcommand{\R}{\mathrm{R}}
\newcommand{\Retr}{\mathrm{Retr}}
\newcommand{\Trans}{\mathrm{Transport}}
\newcommand{\Transbis}[2]{\mathrm{Transport}_{{#1}}({#2})}
\newcommand{\polar}{\mathrm{polar}}
\newcommand{\qf}{\mathrm{qf}}
\newcommand{\Gr}{\mathrm{Gr}}
\newcommand{\St}{\mathrm{St}}
\newcommand{\GLr}{{\mathbb{R}^{r\times r}_*}}
\newcommand{\Precon}{\mathrm{Precon}}
\newcommand{\T}{\mathrm{T}}
\newcommand{\N}{\mathrm{N}}
\newcommand{\HH}{\mathrm{H}}
\newcommand{\VV}{\mathrm{V}}
\newcommand{\M}{\mathcal{M}}
\newcommand{\barM}{\overline{\mathcal{M}}}
\newcommand{\barnabla}{\overline{\nabla}}
\newcommand{\barf}{{\overline{f}}}
\newcommand{\barx}{{\overline{x}}}
\newcommand{\bary}{{\overline{y}}}
\newcommand{\barX}{{\overline{X}}}
\newcommand{\barY}{{\overline{Y}}}
\newcommand{\barg}{{\overline{g}}}
\newcommand{\barxi}{{\overline{\xi}}}
\newcommand{\bareta}{{\overline{\eta}}}
\newcommand{\p}{\mathcal{P}}
\newcommand{\SOn}{{\mathrm{SO}(n)}}
\newcommand{\On}{{\mathrm{O}(n)}}
\newcommand{\Od}{{\mathrm{O}(d)}}
\newcommand{\Op}{{\mathrm{O}(p)}}
\newcommand{\son}{{\mathfrak{so}(n)}}
\newcommand{\SOt}{{\mathrm{SO}(3)}}
\newcommand{\sot}{{\mathfrak{so}(3)}}
\newcommand{\SOtwo}{{\mathrm{SO}(2)}}
\newcommand{\sotwo}{{\mathfrak{so}(2)}}
\newcommand{\So}{{\mathbb{S}^{1}}}
\newcommand{\Stwo}{{\mathbb{S}^{2}}}
\newcommand{\Sn}{{\mathrm{S}^{n-1}}}
\newcommand{\Sone}{{\mathrm{S}^{1}}}
\newcommand{\Sp}{{\mathrm{S}^{2}}}
\newcommand{\Rtt}{{\mathbb{R}^{3\times 3}}}
\newcommand{\Snn}{{\mathbb{S}^{n\times n}}}
\newcommand{\Smm}{{\mathbb{S}^{m\times m}}}
\newcommand{\Smdmd}{{\mathbb{S}^{md\times md}}}
\newcommand{\Spp}{{\mathbb{S}^{p\times p}}}
\newcommand{\Sdd}{{\mathbb{S}^{d\times d}}}
\newcommand{\Srr}{{\mathbb{S}^{r\times r}}}
\newcommand{\Rnn}{{\mathbb{R}^{n\times n}}}
\newcommand{\Rnp}{{\mathbb{R}^{n\times p}}}
\newcommand{\Rmp}{{\mathbb{R}^{m\times p}}}
\newcommand{\Rdp}{{\mathbb{R}^{d\times p}}}
\newcommand{\Rpd}{{\mathbb{R}^{p\times d}}}
\newcommand{\Rmn}{{\mathbb{R}^{m\times n}}}
\newcommand{\Rnm}{\mathbb{R}^{n\times m}}
\newcommand{\Rnr}{\mathbb{R}^{n\times r}}
\newcommand{\Rrr}{\mathbb{R}^{r\times r}}
\newcommand{\Rpp}{\mathbb{R}^{p\times p}}
\newcommand{\Rmm}{{\mathbb{R}^{m\times m}}}
\newcommand{\Rmr}{\mathbb{R}^{m\times r}}
\newcommand{\Rrn}{{\mathbb{R}^{r\times n}}}
\newcommand{\Rd}{{\mathbb{R}^{d}}}
\newcommand{\Rp}{{\mathbb{R}^{p}}}
\newcommand{\Rk}{{\mathbb{R}^{k}}}
\newcommand{\Rdd}{{\mathbb{R}^{d\times d}}}
\newcommand{\RNN}{{\mathbb{R}^{N\times N}}}
\newcommand{\reals}{{\mathbb{R}}}
\newcommand{\complex}{{\mathbb{C}}}
\newcommand{\Rn}{{\mathbb{R}^n}}
\newcommand{\Rm}{{\mathbb{R}^m}}
\newcommand{\Rq}{{\mathbb{R}^q}}
\newcommand{\CN}{{\mathbb{C}^{\,N}}}
\newcommand{\CNN}{{\mathbb{C}^{\ N\times N}}}
\newcommand{\CnotR}{{\mathbb{C}\ \backslash\mathbb{R}}}
\newcommand{\grad}{\mathrm{grad}\,}
\newcommand{\Hess}{\mathrm{Hess}\,}
\newcommand{\vecc}{\mathrm{vec}}
\newcommand{\sign}{\mathrm{sign}}
\newcommand{\diag}{\mathrm{diag}}
\newcommand{\D}{\mathrm{D}}
\newcommand{\dt}{\mathrm{d}t}
\newcommand{\ds}{\mathrm{d}s}
\newcommand{\calM}{\mathcal{M}}
\newcommand{\calN}{\mathcal{N}}
\newcommand{\calC}{\mathcal{C}}
\newcommand{\calD}{\mathcal{D}}
\newcommand{\calE}{\mathcal{E}}
\newcommand{\calF}{\mathcal{F}}
\newcommand{\calX}{\mathcal{X}}
\newcommand{\calL}{\mathcal{L}}
\newcommand{\calU}{\mathcal{U}}
\newcommand{\calH}{\mathcal{H}}
\newcommand{\calO}{\mathcal{O}}
\newcommand{\calT}{\mathcal{T}}
\newcommand{\calA}{\mathcal{A}}
\newcommand{\Id}{\operatorname{Id}} % need to be distinguishable from identity matrix
\newcommand{\rank}{\operatorname{rank}}
\newcommand{\nulll}{\operatorname{null}}
\newcommand{\col}{\operatorname{col}}
\newcommand{\Kmax}{K_{\mathrm{max}}}
\newcommand{\frakF}{\mathfrak{F}}
\newcommand{\frakX}{\mathfrak{X}}


\newcommand{\frobnormbig}[1]{\big\|{#1}\big\|_\mathrm{F}}
%\newcommand{\frobnorm}[1]{\left\|{#1}\right\|_\mathrm{F}}
\newcommand{\opnormsmall}[1]{\|{#1}\|_\mathrm{op}}
\newcommand{\frobnormsmall}[1]{\|{#1}\|_\mathrm{F}}
\newcommand{\smallfrobnorm}[1]{\|{#1}\|_\mathrm{F}}
\newcommand{\norm}[1]{\left\|{#1}\right\|}
\newcommand{\sqnorm}[1]{\left\|{#1}\right\|^2}
%\newcommand{\sqfrobnorm}[1]{\frobnorm{#1}^2}
\newcommand{\sqfrobnormsmall}[1]{\frobnormsmall{#1}^2}
\newcommand{\sqfrobnormbig}[1]{\frobnormbig{#1}^2}

\newcommand{\frobnorm}[2][F]{\left\|{#2}\right\|_\mathrm{#1}}
\newcommand{\sqfrobnorm}[2][F]{\frobnorm[#1]{#2}^2}

\newcommand{\dd}[1]{\frac{\mathrm{d}}{\mathrm{d}#1}}
\newcommand{\dmu}{\mathrm{d}\mu}
\newcommand{\pdd}[1]{\frac{\partial}{\partial #1}}
\newcommand{\TODO}[1]{{\color{red}{[#1]}}}
%\newcommand{\TODOinvisible}[1]{{\color{blue}{#1}}}
\newcommand{\TODOinvisible}[1]{}
%\newcommand{\TODO}[1]{}
\newcommand{\dblquote}[1]{``#1''}
\newcommand{\floor}[1]{\lfloor #1 \rfloor}
\newcommand{\uniform}{\mathrm{Uni}}
\newcommand{\langevin}{\mathrm{Lang}}
\newcommand{\langout}{\mathrm{LangUni}}
\newcommand{\Zeq}{{Z_\mathrm{eq}}}
\newcommand{\feq}{{f_\mathrm{eq}}}
\newcommand{\XM}{\mathfrak{X}(\M)}
\newcommand{\FM}{\mathfrak{F}(\M)}
\newcommand{\length}{\operatorname{length}}
\newcommand{\bfR}{\mathbf{R}}
\newcommand{\bfxi}{{\bm{\xi}}}
\newcommand{\bfX}{{\mathbf{X}}}
\newcommand{\bfY}{{\mathbf{Y}}}
\newcommand{\bfeta}{{\bm{\eta}}}
\newcommand{\bfF}{\mathbf{F}}
\newcommand{\bfOmega}{\boldsymbol{\Omega}}
\newcommand{\bftheta}{{\bm{\theta}}}
\newcommand{\lambdamax}{\lambda_\mathrm{max}}
\newcommand{\lambdamin}{\lambda_\mathrm{min}}


\newcommand{\ddiag}{\mathrm{ddiag}}
\newcommand{\dist}{\mathrm{dist}}
\newcommand{\embdist}{\mathrm{embdist}}
\newcommand{\var}{\mathrm{var}}
\newcommand{\Cov}{\mathbf{C}}
\newcommand{\Covm}{C}
\newcommand{\FIM}{\mathbf{F}}
\newcommand{\FIMm}{F}
\newcommand{\Rmle}{{\hat\bfR_\mle}}
%\newcommand{\diag}{\mathbf{\mathrm{diag}}}
\newcommand{\Rmoptensor}{\mathbf{{R_\mathrm{m}}}}
\newcommand{\barRmoptensor}{\mathbf{{\bar{R}_\mathrm{m}}}}
\newcommand{\Rmop}{R_\mathrm{m}}
\newcommand{\barRmop}{\bar{R}_\mathrm{m}}
\newcommand{\Rcurv}{\mathcal{R}}
\newcommand{\barRcurv}{\bar{\mathcal{R}}}

\newcommand{\ith}{$i$th }
\newcommand{\jth}{$j$th }
\newcommand{\kth}{$k$th }
\newcommand{\ellth}{$\ell$th }

\newcommand{\expp}[1]{{\exp\!\big({#1}\big)}}


%\definecolor{darkgreen}{rgb}{0,.5,0}

\newtheorem{theorem}{Theorem}[section] % Comment if SIOPT format
\newtheorem{lemma}[theorem]{Lemma} % Comment if SIOPT format
\newtheorem{proposition}[theorem]{Proposition} % Comment if SIOPT format
\newtheorem{corollary}[theorem]{Corollary} % Comment if SIOPT format
\newtheorem{assumption}[theorem]{Assumption}
\newtheorem{example}[theorem]{Example}%[section]
\newtheorem{definition}[theorem]{Definition} % Comment if SIOPT format
\newtheorem{remark}[theorem]{Remark}

%\newenvironment{proof}[1][Proof]{\begin{trivlist}
%\item[\hskip \labelsep {\bfseries #1}]}{\end{trivlist}}
%
%\newenvironment{definition}[1][Definition]{\begin{trivlist}
%\item[\hskip \labelsep {\bfseries #1}]}{\end{trivlist}}
%\newenvironment{example}[1][Example]{\begin{trivlist}
%\item[\hskip \labelsep {\bfseries #1}]}{\end{trivlist}}
%\newenvironment{remark}[1][Remark]{\begin{trivlist}
%\item[\hskip \labelsep {\bfseries #1}]}{\end{trivlist}}
%
%\newcommand{\qed}{\nobreak \ifvmode \relax \else
%      \ifdim\lastskip<1.5em \hskip-\lastskip
%      \hskip1.5em plus0em minus0.5em \fi \nobreak
%      \vrule height0.75em width0.5em depth0.25em\fi}


\newcommand{\minimize}{\operatorname{minimize}}

\newcommand{\st}{\textrm{ subject to }}

\newcommand{\E}{\mathbb{E}}

\title{Formalism for autocorrelation derivations}

%\author{
%Nicolas Boumal\thanks{Princeton University, Mathematics Department and PACM, \texttt{nboumal@math.princeton.edu}}  \and Vladislav Voroninski\thanks{Helm.ai} \and Afonso S.\ Bandeira\thanks{Department of Mathematics and Center for Data Science, Courant Institute of Mathematical Sciences, New York University, \texttt{bandeira@cims.nyu.edu}}
%}
\author{(T,N)B}
\date{February 15, 2019}


\begin{document}

\maketitle


Let $x_{(1)}, \ldots, x_{(|s|)}$ denote the (independent) realizations of the random signal $x$ in the observation $y$, starting at (deterministic) positions $s_{(1)}, \ldots, s_{(|s|)}$. Let $I_{ij}$ be the indicator variable for whether position $i$ is in the support of occurrence $j$, that is, it is one if $i$ is in $\{s_{(j)}, \ldots, s_{(j)}+L-1\}$, and zero otherwise. Then,
\begin{align}
y[i] & = \sum_{j = 1}^{|s|} I_{ij} x_{(j)}[i-s_{(j)}] + \varepsilon[i].
\label{eq:explicityiindicators}
\end{align}
This gives a simple expression for the first autocorrelation of $y$. Indeed,
\begin{align}
a_y^1 & = \E_y\left\{ \frac{1}{N} \sum_{i = 0}^{N-1} y[i] \right\} \\
& = \frac{1}{N} \E_{x_{(1)}, \ldots, x_{(|s|)}, \varepsilon}\left\{ \sum_{i = 0}^{N-1} \sum_{j = 1}^{|s|} I_{ij} x_{(j)}[i-s_{(j)}] + \varepsilon[i] \right\}.
\end{align}
Now switch the sums over $i$ and $j$, and observe that $I_{ij}$ is zero unless $i = s_{(j)} + t$ for $t$ in the range $0, \ldots, L-1$. Hence,
\begin{align}
a_y^1 & = \frac{1}{N} \sum_{j = 1}^{|s|} \E_{x_{(j)}}\left\{ \sum_{t = 0}^{L-1} x_{(j)}[t]\right\} + \frac{1}{N} \E_\varepsilon\left\{ \sum_{i=0}^{N-1} \varepsilon[i]\right\}.
\end{align}
Since the noise has zero mean and $x_{(1)}, \ldots, x_{(|s|)}$ are independent and all distributed as $x$, we further find:
\begin{align}
a_y^1 & = \frac{|s|L}{N} a_x^1 = \gamma a_x^1.
\end{align}

To address the second-order moments, we resort to the separation conditions. First, consider this expression:
\begin{align*}
N \cdot a_y^2[\ell] & = \E_y\left\{ \sum_{i = 0}^{N-\ell-1} y[i] y[i+\ell] \right\} \\
& = \sum_{i = 0}^{N-\ell-1} \E_{x_{(1)}, \ldots, x_{(|s|)}, \varepsilon}\Bigg\{ \left( \sum_{j = 1}^{|s|} I_{ij} x_{(j)}[i-s_{(j)}] + \varepsilon[i] \right) \cdot \\
& \qquad \qquad \qquad \qquad \qquad  \left( \sum_{j' = 1}^{|s|} I_{i+\ell,j'} x_{(j')}[i+\ell-s_{(j')}] + \varepsilon[i+\ell] \right)  \Bigg\} \\
& = \sum_{i = 0}^{N-\ell-1} \E_{x_{(1)}, \ldots, x_{(|s|), \varepsilon}}\Bigg\{ \sum_{j = 1}^{|s|} \sum_{j' = 1}^{|s|} I_{ij}  I_{i+\ell,j'} x_{(j)}[i-s_{(j)}]  x_{(j')}[i+\ell-s_{(j')}] \\
& \qquad \qquad \qquad \qquad \qquad \qquad + \sum_{j = 1}^{|s|} I_{ij} x_{(j)}[i-s_{(j)}] \varepsilon[i+\ell] \\
& \qquad \qquad \qquad \qquad \qquad \qquad + \sum_{j' = 1}^{|s|} I_{i+\ell,j'} x_{(j')}[i+\ell-s_{(j')}] \varepsilon[i] \\
& \qquad \qquad \qquad \qquad \qquad \qquad + \varepsilon[i] \varepsilon[i + \ell] \Bigg\}.
\end{align*}
The cross-terms vanish in expectation since $\varepsilon$ is zero mean and independent from the signal occurrences. The last term vanishes in expectation unless $\ell = 0$ since distinct entries of $\varepsilon$ are independent. For $\ell = 0$, $\E\{\varepsilon[i]^2\} = \sigma^2$. Finally, using the separation property, observe that if $I_{ij}  I_{i+\ell,j'}$ is nonzero, then it is equal to one, $j = j'$ and $i = s_{(j)} + t$ for some $t$ in $0, \ldots, L-\ell-1$. Then, switch the order of summations to get
\begin{align}
N \cdot a_y^2[\ell] & = \sum_{j=1}^{|s|} \E_{x_{(j)}}\left\{ \sum_{t = 0}^{L-\ell-1} x_{(j)}[t] x_{(j)}[t+\ell] \right\} + (N-\ell)\sigma^2 \delta[\ell],
\end{align}
where $\delta[0] = 1$ and $\delta[\ell \neq 0] = 0$. Since each $x_{(j)}$ is distributed as $x$, they all have the same autocorrelations as $x$ and we finally get
\begin{align}
a_y^2[\ell] & = \gamma a_x^2[\ell] + \frac{N-\ell}{N}\sigma^2 \delta[\ell] = \gamma a_x^2[\ell] + \sigma^2 \delta[\ell].
\end{align}

We now turn to the third-order autocorrelations. These involve the sum
\begin{align}
\sum_{i=0}^{N-\max(\ell_1, \ell_2)-1} y[i] y[i+\ell_1] y[i+\ell_2].
\end{align}
Using~\eqref{eq:explicityiindicators}, we find that this quantity can be expressed as a sum of eight terms:
% \sum_{j = 1}^{|s|} I_{ij} x_{(j)}[i-s_{(j)}] + \varepsilon[i]
\begin{enumerate}
	\item $\sum_i \sum_{j,j',j'' = 1}^{|s|} I_{ij} I_{i+\ell_1, j'} I_{i+\ell_2,j''} x_{(j)}[i-s_{(j)}] x_{(j')}[i+\ell_1-s_{(j')}] x_{(j'')}[i+\ell_2-s_{(j'')}]$
	\item $\sum_i \sum_{j,j' = 1}^{|s|} I_{ij} I_{i+\ell_1, j'} x_{(j)}[i-s_{(j)}] x_{(j')}[i+\ell_1-s_{(j')}] \varepsilon[i+\ell_2]$
	\item $\sum_i \sum_{j,j'' = 1}^{|s|} I_{ij} I_{i+\ell_2,j''} x_{(j)}[i-s_{(j)}] \varepsilon[i+\ell_1] x_{(j'')}[i+\ell_2-s_{(j'')}]$
	\item $\sum_i \sum_{j',j'' = 1}^{|s|} I_{i+\ell_1, j'} I_{i+\ell_2,j''} \varepsilon[i] x_{(j')}[i+\ell_1-s_{(j')}] x_{(j'')}[i+\ell_2-s_{(j'')}]$
	\item $\sum_i \sum_{j = 1}^{|s|} I_{ij} x_{(j)}[i-s_{(j)}] \varepsilon[i+\ell_1] \varepsilon[i+\ell_2]$
	\item $\sum_i \sum_{j' = 1}^{|s|} I_{i+\ell_1, j'} \varepsilon[i] x_{(j')}[i+\ell_1-s_{(j')}] \varepsilon[i+\ell_2]$
	\item $\sum_i \sum_{j'' = 1}^{|s|} I_{i+\ell_2,j''} \varepsilon[i] \varepsilon[i+\ell_1] x_{(j'')}[i+\ell_2-s_{(j'')}]$
	\item $\sum_i \varepsilon[i] \varepsilon[i+\ell_1] \varepsilon[i+\ell_2]$
\end{enumerate}
Terms 2--4 and 8 vanish in expectation since odd moments of centered Gaussian variables are zero. For the first term, we use the fact that the separation condition implies
\begin{multline}
I_{ij} I_{i+\ell_1, j'} I_{i+\ell_2,j''} = 1 \iff \\ j=j'=j'' \textrm{ and } i = s_{(j)} + t \textrm{ with } t \in \{ 0, \ldots L-\max(\ell_1, \ell_2)-1 \}.
\end{multline}
(Otherwise, the product of indicators is zero.) This allows to reduce the summations over $j,j',j''$ to a single sum over $j$. Then, witching the order of summation with $i$, we get that the first term is equal to
\begin{align}
\sum_{j=1}^{|s|} \sum_{t=0}^{L-\max(\ell_1, \ell_2)-1} x_{(j)}[t] x_{(j)}[t+\ell_1] x_{(j)}[t+\ell_2].
\end{align}
In expectation over the realizations $x_{(j)}$, using again that they are i.i.d.\ with the same distribution as $x$, this first term yields $|s|L a_x^3[\ell_1, \ell_2]$. Now consider the fifth term. Taking expectation against $\varepsilon$ yields
\begin{align}
\sum_{i=0}^{N-\max(\ell_1, \ell_2)-1} \sum_{j = 1}^{|s|} I_{ij} x_{(j)}[i-s_{(j)}] \sigma^2 \delta[\ell_1 - \ell_2].
\end{align}
Switch the order of summation over $i$ and $j$ again to get
\begin{align}
\sigma^2 \delta[\ell_1 - \ell_2] \sum_{j = 1}^{|s|} \sum_{t=0}^{L-1} x_{(j)}[t].
\end{align}
Now taking expectation against the signal occurrences yields $|s|L \sigma^2 a_x^1 \delta[\ell_1 - \ell_2]$. A similar reasoning for terms 6 and 7 yields this final formula for the third-order autocorrelations of $y$:
\begin{align}
a_y^3[\ell_1, \ell_2] & = \gamma a_x^3[\ell_1, \ell_2] + \gamma \sigma^2 a_x^1 \left( \delta[\ell_1] + \delta[\ell_2] + \delta[\ell_1 - \ell_2] \right).
\end{align}

\end{document}
